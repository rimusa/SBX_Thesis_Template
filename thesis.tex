% Note that you need to use LuaLaTeX for this to even compile
\documentclass[11pt,a4paper]{DL}


% This lets us offload the preamble so our working space is cleaner
\usepackage{import}

% Write the paths to your folders here
\newcommand{\PreamblePath}{./00_Preamble}  % We don't want the preamble to clutter everything
\newcommand{\PartsPath}{./01_Parts}


% These are packages that Ricardo uses all of the time and that he considers useful to add by default
% Feel free to modify it if it doesn't suit your needs
% You can find the latest version of this file here:
% https://github.com/rimusa/useful_stuff/blob/main/latex/packages.tex
\import{\PreamblePath}{boilerplate-packages}


% Additional packages should be added either below or in the imported file
\import{\PreamblePath}{imported-packages}


% Special variables for your thesis
% You'll need to modify this file to set your name, the title of the thesis, etc.
\import{\PreamblePath}{thesis-variables}


% Import your bib files here
\import{\PreamblePath}{bib-files}


% Here we finally begin the document
\begin{document}

% Everything in the thesis lives in wrapper files, found in the 01_Parts folder
% They should let you work on individual parts of your thesis without having to think of the others
% They assume your thesis is roughly separarated as follows:
%  - Title Pages
%  - Preface
%  - Index
%  - Kappa
%  - Papers
%  - Appendices
%  - Bibliography

\maketitle
\frontmatter

\begin{abstract}
    Add a short description of your thesis here (around one or two pages?).
\end{abstract}
\begin{sammanfattning}
    You should add a version of your abstract in Swedish here.
\end{sammanfattning}
\input{02_Preface/acknowledgements.tex}

\tableofcontents
\mainmatter

\part{Kappa} 
% It is generated as a separate page with only the title
\label{part:kappa}

% This section should provide a general introduction to your work
% The chapters here are a suggestion, feel free to reorganize and rename things around

\chapter{Introduction}
\label{chapter:introduction}
\section{Research questions and contributions}

Present your research questions and how they relate to your thesis-


\section{Overview of publications}

The following publications are included in this thesis:
\begin{enumerate}
    \item Paper 1
    \item Paper 2
\end{enumerate}


\section{Structure of the thesis}

Talk about what structure your thesis will follow.

\chapter{Background}
\label{chapter:background}
%\section{What Others Did}

Give context to your work.

\chapter{Methods and Resources}
\label{chapter:methods}
%How we did what what we did.

\chapter{Discussion}
\label{chapter:discussion}
%What is the impact of our work.

% Add your papers here as if they were chapters
\part{Papers}
\label{part:papers}

\chapter{Test Chapter}
AAAA


\chapter{Your Paper}


\textbf{Abstract.}
Your abstract goes here.

\section{Introduction}
Here goes the intro to your paper.

\section{ETC.}
You keep adding sections as you need.

\section*{Acknowledgements}
Unnumbered sections such as the ones from the acknowledgements can be used like this.

% Use this if you want to have appendices follow the paper they originally appeared in
\begin{subappendices}

\section{An Appendix}
We can have appendices with letters to mark them instead of running numbers or no numeration at all.

    
\end{subappendices}


% Use this for global appendices
\part{Appendices}
\label{part:appendices}

\appendix
\markboth{}{}      
\extramarks{}{} 

\input{05_Appendices/example}


% We print the bibliography *after* the appendices
% This bit of code comes from Felix
\defbibnote{referencedoc}{\markboth{References}{References}}
\cleardoublepage
\addcontentsline{toc}{chapter}{\numberline{}\refname}
\markboth{\refname}{\refname}
\extramarks{\refname}{\refname}
\chptitle{\refname}
\printbibliography[title={References}, prenote=referencedoc]

\backmatter

\end{document}
